\documentclass{article}
\usepackage[final]{nips}

\usepackage{amsthm}
\usepackage{amsmath}
\usepackage{amssymb}
\usepackage{graphicx}
\usepackage{mathtools}
\usepackage{enumerate}
\usepackage{enumitem}
\usepackage{footnote}
\usepackage{float}
\usepackage{xspace}
\usepackage{multirow}
\usepackage{nicefrac}
\usepackage{wrapfig}
\usepackage{framed}
\usepackage{url}
\usepackage[colorlinks=true, linkcolor=blue, citecolor=blue]{hyperref}
\usepackage[ruled, vlined]{algorithm2e}
\usepackage{blkarray}
\PassOptionsToPackage{round}{natbib}

\usepackage{tikz}
\usetikzlibrary{arrows,chains,matrix,positioning,scopes}

\newcommand{\push}{\hspace{0pt plus 1 filll} }	

\newtheorem{lemma}{Lemma}
\newtheorem{theorem}{Theorem}
\newtheorem{cor}{Corollary}
\newtheorem{remark}{Remark}
\newtheorem{prop}{Proposition}
\newtheorem{property}{Property}
\newtheorem{definition}{Definition}
\newtheorem{assumption}{Assumption}

\newcommand{\calA}{{\mathcal{A}}}
\newcommand{\calH}{{\mathcal{H}}}
\newcommand{\calL}{{\mathcal{L}}}
\newcommand{\calX}{{\mathcal{X}}}
\newcommand{\calY}{{\mathcal{Y}}}
\newcommand{\calS}{{\mathcal{S}}}
\newcommand{\calI}{{\mathcal{I}}}
\newcommand{\calD}{{\mathcal{D}}}
\newcommand{\calK}{{\mathcal{K}}}
\newcommand{\calE}{{\mathcal{E}}}
\newcommand{\calR}{{\mathcal{R}}}
\newcommand{\calT}{{\mathcal{T}}}
\newcommand{\calP}{{\mathcal{P}}}
\newcommand{\calZ}{{\mathcal{Z}}}
\newcommand{\calM}{{\mathcal{M}}}
\newcommand{\calN}{{\mathcal{N}}}
\newcommand{\calF}{{\mathcal{F}}}
\newcommand{\calV}{{\mathcal{V}}}
\newcommand{\calQ}{{\mathcal{Q}}}
\newcommand{\ips}{\wh{r}}
\newcommand{\whpi}{\wh{\pi}}
\newcommand{\whE}{\wh{\E}}
\newcommand{\whV}{\wh{V}}
\newcommand{\reg}{{\mathcal{R}}}
\newcommand{\breg}{{\mathcal{\bar{R}}}}
\newcommand{\hmu}{\wh{\mu}}
\newcommand{\tmu}{\wt{\mu}}
\newcommand{\one}{\boldsymbol{1}}
\newcommand{\loss}{\ell}
\newcommand{\hloss}{\wh{\ell}}
\newcommand{\bloss}{\bar{\ell}}
\newcommand{\tloss}{\wt{\ell}}
\newcommand{\htheta}{\wh{\theta}}

\newcommand{\bs}{{\boldsymbol{s}}}
\newcommand{\bz}{{\boldsymbol{z}}}
\newcommand{\bx}{{\boldsymbol{x}}}
\newcommand{\br}{{\boldsymbol{r}}}
\newcommand{\bX}{{\boldsymbol{X}}}
\newcommand{\bu}{{\boldsymbol{u}}}
\newcommand{\by}{{\boldsymbol{y}}}
\newcommand{\bY}{{\boldsymbol{Y}}}
\newcommand{\bg}{{\boldsymbol{g}}}
\newcommand{\ba}{{\boldsymbol{a}}}
\newcommand{\be}{{\boldsymbol{e}}}
\newcommand{\bq}{{\boldsymbol{q}}}
\newcommand{\bp}{{\boldsymbol{p}}}
\newcommand{\bZ}{{\boldsymbol{Z}}}
\newcommand{\bS}{{\boldsymbol{S}}}
\newcommand{\bw}{{\boldsymbol{w}}}
\newcommand{\bW}{{\boldsymbol{W}}}
\newcommand{\bU}{{\boldsymbol{U}}}
\newcommand{\bv}{{\boldsymbol{v}}}
\newcommand{\bzero}{{\boldsymbol{0}}}
\newcommand{\beps}{{\boldsymbol{\epsilon}}}

\newcommand{\blue}[1]{{\color{blue}#1}}

\DeclareMathOperator*{\argmin}{argmin}
\DeclareMathOperator*{\argmax}{argmax}
%\DeclareMathOperator*{\liminf}{liminf}
%\DeclareMathOperator*{\limsup}{limsup}
\DeclareMathOperator*{\range}{range}
\DeclareMathOperator*{\mydet}{det_{+}}

\newcommand{\field}[1]{\mathbb{#1}}
\newcommand{\fY}{\field{Y}}
\newcommand{\fX}{\field{X}}
\newcommand{\fH}{\field{H}}
\newcommand{\fR}{\field{R}}
\newcommand{\fB}{\field{B}}
\newcommand{\fS}{\field{S}}
\newcommand{\fN}{\field{N}}
\newcommand{\E}{\field{E}}
\renewcommand{\P}{\field{P}}

\newcommand{\theset}[2]{ \left\{ {#1} \,:\, {#2} \right\} }
%\newcommand{\inner}[1]{ \left\langle {#1} \right\rangle }
\newcommand{\Ind}[1]{ \field{I}{\{{#1}\}} }
\newcommand{\eye}[1]{ \boldsymbol{I}_{#1} }
\newcommand{\norm}[1]{\left\|{#1}\right\|}
%\newcommand{\trace}[1]{\text{tr}\left({#1}\right)}
\newcommand{\trace}[1]{\textsc{tr}({#1})}
\newcommand{\diag}[1]{\mathrm{diag}\!\left\{{#1}\right\}}
\newcommand{\RE}{{\text{\rm RE}}}
\newcommand{\KL}{{\text{\rm KL}}}
\newcommand{\LCB}{{\text{\rm LCB}}}
\newcommand{\Reg}{{\text{\rm Reg}}}
\newcommand{\Rel}{{\text{\rm Rel}}}
%\newcommand{\ERM}{{\text{\rm ERM}}\xspace}

\newcommand{\defeq}{\stackrel{\rm def}{=}}
\newcommand{\sgn}{\mbox{\sc sgn}}
\newcommand{\scI}{\mathcal{I}}
\newcommand{\scO}{\mathcal{O}}
\newcommand{\scN}{\mathcal{N}}

\newcommand{\dt}{\displaystyle}
\renewcommand{\ss}{\subseteq}
\newcommand{\wh}{\widehat}
\newcommand{\wt}{\widetilde}
\newcommand{\ve}{\varepsilon}
\newcommand{\hlambda}{\wh{\lambda}}
\newcommand{\yhat}{\wh{y}}
\newcommand{\pred}{\yhat}

\newcommand{\hDelta}{\wh{\Delta}}
\newcommand{\hdelta}{\wh{\delta}}
\newcommand{\spin}{\{-1,+1\}}

\newcommand{\paren}[1]{\left({#1}\right)}
\newcommand{\brackets}[1]{\left[{#1}\right]}
\newcommand{\braces}[1]{\left\{{#1}\right\}}

\newcommand{\normt}[1]{\norm{#1}_{t}}
\newcommand{\dualnormt}[1]{\norm{#1}_{t,*}}

\newcommand{\order}{\ensuremath{\mathcal{O}}}
\newcommand{\otil}{\ensuremath{\widetilde{\mathcal{O}}}}
\newcommand{\risk}{{\text{\rm Risk}}}
\newcommand{\iid}{{\text{\rm iid}}}
\newcommand{\seq}{{\text{\rm seq}}}
\newcommand{\iidV}{\calV^\iid}
\newcommand{\seqV}{\calV^\seq}
\newcommand{\poly}{{\text{\rm poly}}}
\newcommand{\sign}{{\text{\rm sign}}}
\newcommand{\ERM}{\pred_{{\text{\rm ERM}}}}
\newcommand{\iidRad}{\calR^{\iid}}
\newcommand{\iidCRad}{\wh{\calR}^{\iid}}
\newcommand{\seqRad}{\calR^{\seq}}
\newcommand{\seqCRad}{\wh{\calR}^{\seq}}
\newcommand{\VC}{{\text{\rm VCdim}}}
\newcommand{\Pdim}{{\text{\rm Pdim}}}
\newcommand{\Ldim}{{\text{\rm Ldim}}}
\newcommand{\fat}{{\text{\rm fat}}}
\newcommand{\sfat}{{\text{\rm sfat}}}
\newcommand{\vol}{{\text{\rm Vol}}}
\newcommand{\Holder}{{H{\"o}lder}\xspace}

%%%%  brackets
\newcommand{\inner}[2]{\left\langle #1,#2 \right\rangle}
\newcommand{\minimax}[1]{\left\llangle #1 \right\rrangle}
\newcommand{\rbr}[1]{\left(#1\right)}
\newcommand{\sbr}[1]{\left[#1\right]}
\newcommand{\cbr}[1]{\left\{#1\right\}}
\newcommand{\nbr}[1]{\left\|#1\right\|}
\newcommand{\abr}[1]{\left|#1\right|}

\DeclareFontFamily{OMX}{MnSymbolE}{}
\DeclareFontShape{OMX}{MnSymbolE}{m}{n}{
    <-6>  MnSymbolE5
   <6-7>  MnSymbolE6
   <7-8>  MnSymbolE7
   <8-9>  MnSymbolE8
   <9-10> MnSymbolE9
  <10-12> MnSymbolE10
  <12->   MnSymbolE12}{}
\DeclareSymbolFont{mnlargesymbols}{OMX}{MnSymbolE}{m}{n}
\SetSymbolFont{mnlargesymbols}{bold}{OMX}{MnSymbolE}{b}{n}
\DeclareMathDelimiter{\llangle}{\mathopen}{mnlargesymbols}{'164}{mnlargesymbols}{'164}
\DeclareMathDelimiter{\rrangle}{\mathclose}{mnlargesymbols}{'171}{mnlargesymbols}{'171}

\usepackage{prettyref}
\newcommand{\pref}[1]{\prettyref{#1}}
\newcommand{\pfref}[1]{Proof of \prettyref{#1}}
\newcommand{\savehyperref}[2]{\texorpdfstring{\hyperref[#1]{#2}}{#2}}
\newrefformat{eq}{\savehyperref{#1}{\textup{(\ref*{#1})}}}
\newrefformat{eqn}{\savehyperref{#1}{Equation~\eqref{#1}}}
\newrefformat{lem}{\savehyperref{#1}{Lemma~\ref*{#1}}}
\newrefformat{def}{\savehyperref{#1}{Definition~\ref*{#1}}}
\newrefformat{line}{\savehyperref{#1}{line~\ref*{#1}}}
\newrefformat{thm}{\savehyperref{#1}{Theorem~\ref*{#1}}}
\newrefformat{corr}{\savehyperref{#1}{Corollary~\ref*{#1}}}
\newrefformat{cor}{\savehyperref{#1}{Corollary~\ref*{#1}}}
\newrefformat{sec}{\savehyperref{#1}{Section~\ref*{#1}}}
\newrefformat{app}{\savehyperref{#1}{Appendix~\ref*{#1}}}
\newrefformat{assum}{\savehyperref{#1}{Assumption~\ref*{#1}}}
\newrefformat{ex}{\savehyperref{#1}{Example~\ref*{#1}}}
\newrefformat{fig}{\savehyperref{#1}{Figure~\ref*{#1}}}
\newrefformat{alg}{\savehyperref{#1}{Algorithm~\ref*{#1}}}
\newrefformat{rem}{\savehyperref{#1}{Remark~\ref*{#1}}}
\newrefformat{conj}{\savehyperref{#1}{Conjecture~\ref*{#1}}}
\newrefformat{prop}{\savehyperref{#1}{Proposition~\ref*{#1}}}
\newrefformat{proto}{\savehyperref{#1}{Protocol~\ref*{#1}}}
\newrefformat{prob}{\savehyperref{#1}{Problem~\ref*{#1}}}
\newrefformat{claim}{\savehyperref{#1}{Claim~\ref*{#1}}}
\newrefformat{que}{\savehyperref{#1}{Question~\ref*{#1}}}

\title{CSCI 678: Theoretical Machine Learning \\ Homework 2 \\ {\small Fall 2024, Instructor: Haipeng Luo}}  

\begin{document}
\maketitle

\textit{This homework is due on {\bf 10/13, 11:59pm}. 
See course website for more instructions on finishing and submitting your homework as well as the late policy. Total points: \blue{50}}\\

\begin{enumerate}[leftmargin=*,align=left]
\item 
({\bf Pseudo-dimension and fat-shattering dimension})
For a function $f: [0,1] \rightarrow [-1,1]$, define its total variation $V(f)$ as 
\[
V(f) = \sup_{\substack{1 \leq m \in \mathbf{Z}_+ \\ 0= x_0 < x_1 < \cdots < x_m=1}}\;\sum_{j=1}^m |f(x_j) - f(x_{j-1})|,
\]
which, intuitively, measures how much the function varies on the interval $[0,1]$.
Now, consider the function class $\calF=\cbr{f: [0,1] \rightarrow [-1,1] \;|\; V(f) \leq B}$ for some constant $B > 0$.

\begin{enumerate}[leftmargin=*,align=left]
\item (\blue{4pts})  Prove that the Pseudo-dimension of $\calF$ is infinity. \\


\newpage
\item Follow the two steps below to prove that the fat-shattering dimension of $\calF$ at scale $\alpha \leq 1$ is
\[
\fat(\calF, \alpha) = 1 + \left\lfloor \frac{B}{\alpha} \right\rfloor.
\]

\begin{enumerate}[leftmargin=*,align=left]
\item (\blue{4pts})  For $n \leq 1 + \frac{B}{\alpha}$, construct a sequence of $n$ pairs $(x_1, y_1), \ldots, (x_n, y_n) \in [0,1] \times [-1,1]$, such that for any labeling $s_1, \ldots, s_n \in \cbr{-1,+1}$, there exists $f\in\calF$ with $s_t(f(x_t) - y_t) \geq \alpha/2$ for all $t = 1, \ldots, n$. (This shows $\fat(\calF, \alpha) \geq 1 + \left\lfloor \frac{B}{\alpha} \right\rfloor$.)\\


\vspace{5pt}
\item (\blue{5pts})  For any $n > 1 + \frac{B}{\alpha}$ and any sequence of $n$ pairs $(x_1, y_1), \ldots, (x_n, y_n) \in [0,1] \times [-1,1]$ with $x_1 < x_2 < \cdots < x_n$, show that if $f: [0,1]\rightarrow [-1,1]$ is such that $s_t(f(x_t) - y_t) \geq \alpha/2$ for all $t = 1, \ldots, n$ where
\[
s_1 = -1, s_2 = +1, s_3 = -1, s_4=+1, \ldots,
\]
and $g: [0,1]\rightarrow [-1,1]$ is such that $s_t(g(x_t) - y_t) \geq \alpha/2$ for all $t = 1, \ldots, n$ where
\[
s_1 = +1, s_2 = -1, s_3 = +1, s_4=-1, \ldots,
\]
then we must have $V(f)+V(g) > 2B$. 
(Convince yourself that this implies $\fat(\calF, \alpha) \leq 1 + \left\lfloor \frac{B}{\alpha} \right\rfloor$.)\\

\end{enumerate}

\end{enumerate}

\newpage
\item
({\bf Zero-covering number and shattering})
Consider a class of binary predictors $\calF \subset \cbr{-1,+1}^\calX$.
The concept of zero-covering number $\calN_0(\calF|_\bx)$ given an $\calX$-valued tree $\bx$ of depth $n$ is analogous to $|\calF|_{x_{1:n}}|$, the cardinality of the projection of $\calF$ on a dataset $x_{1:n}$ (in the statistical learning setting).
However, there are some subtle differences between them.
In particular, while $|\calF|_{x_{1:n}}| = 2^n$ is equivalent to $x_{1:n}$ being shattered by $\calF$, $\calN_0(\calF|_\bx) = 2^n$ is \textit{not} equivalent to $\bx$ being shattered by $\calF$.
In this problem, you will explore why this is case. 
(Understanding what the questions below are asking you to do is already a good test to your understanding of the related concepts.)

\begin{enumerate}[leftmargin=*,align=left]

\vspace{5pt}
\item (\blue{4pts}) 
Prove that if $\calF$ shatters $\bx$, then we indeed have $\calN_0(\calF|_\bx) = 2^n$.
(Recall that $\calN_0(\calF|_\bx) \leq 2^n$ is always true, so this is really asking you to show $\calN_0(\calF|_\bx) \geq 2^n$.) \\

\vspace{5pt}
\item (\blue{4pts}) 
Next, prove that $\calN_0(\calF|_\bx) = 2^n$ does not necessarily mean that $\calF$ shatters $\bx$. 
Hint: consider a tree $\bx$ with depth $n$ being the VC-dimension of $\calF$ and the leftmost path consisting of $n$ points that are shattered by $\calF$ (in the statistical learning sense). \\


\vspace{5pt}
\item (\blue{4pts}) 
Finally, prove that if $\calN_0(\calF|_\bx) = 2^n$, then there must exist a tree $\bx'$ of depth $n$ that is shattered by $\calF$. Hint: use Theorem 1 of Lecture 6, that is, the online analogue of Sauer's lemma. 
(Note that combining (a) and (c), we have 
\[
\Ldim(\calF) = \max\cbr{n: \max_{\bx \text{ of depth $n$}} \calN_0(\calF |_\bx)=2^n},
\] 
which is analogous to $\VC(\calF) = \max\cbr{n: \max_{x_{1:n}} |\calF |_{x_{1:n}}|=2^n}$.) \\


\end{enumerate}

\newpage
\item
({\bf Littlestone dimension})
Consider $\calX = \fR^d$ and the class 
\begin{equation*}\label{eqn:simple_class}
\calF = \cbr{f_{\theta,b}(x) = \begin{cases}
+1, &\text{if $\inner{\theta}{x}+b=0$} \\
-1, &\text{else}
\end{cases} \;\bigg\rvert\; 0 \neq \theta \in \fR^d, b \in \fR}.
\end{equation*}
which is a generalization of the simple class Eq.~(5) in Lecture 5 from one dimension to general dimension.
In words, it classifies all the points residing in the hyperplane $\inner{\theta}{x}+b=0$ as $+1$, and everything else as $-1$.
Follow the steps below to show $\Ldim(\calF) = d$.

\begin{enumerate}[leftmargin=*,align=left]
\vspace{5pt}
\item (\blue{3pts}) 
Construct a set of $d$ points $x_1, \ldots, x_d \in \fR^d$ that can be shattered by $\calF$ (in the statistical learning sense), which shows $d \leq \VC(\calF) \leq \Ldim(\calF)$. \\


\vspace{5pt}
\item (\blue{4pts}) 
For $d = 2$, show that no tree $\bx$ of depth $3$ can be shattered by $\calF$.
Hint: consider different cases for the three points on the rightmost path of $\bx$: are they collinear (that is, on the same line)? are some of them identical? \\


\vspace{5pt}
\item (\blue{8pts}) 
Generalize the idea from the last question to show that for any dimension $d$, no tree of depth $d+1$ can be shattered by $\calF$, which shows $\Ldim(\calF) \leq d$. 
Hint: a set of $n$ points $x_1, \ldots, x_n \in \fR^d$ are \textit{affinely} dependent if the following $n-1$ points are linearly dependent: $x_1 - x_n, x_2-x_n, \ldots, x_{n-1}-x_{n}$; 
convince yourself that two points being affinely dependent if and only if they are identical, and three points being affinely dependent if and only if they are collinear. \\


\end{enumerate}


\newpage
\item
({\bf Lower bound for online classification})
In this exercise you will prove $\seqV(\calF, n) \geq \sqrt{\frac{d}{8n}}$ where $d = \Ldim(\calF) \leq n$.
For simplicity, we will further assume that $n$ is a multiple of $d$.
The construction of the environment is as follows.
The labels $y_1, \ldots, y_n$ are i.i.d. Rademacher random variables.
To define the example $x_1, \ldots, x_n$,
we divide the entire $n$ rounds evenly into $d$ epochs,
where epoch $k$ contains rounds $n(k-1)/d+1, \ldots, nk/d$.
On the same epoch, $x_t$ stays the same.
Specifically, let 
$
\epsilon_k = \sign\rbr{\sum_{t \in \text{epoch $k$}} y_t}
$
be the majority vote of the true labels in epoch $k$, that is,
\[
\epsilon_k = \begin{cases}
+1, &\text{ if $\sum_{t \in \text{epoch $k$}} y_t \geq 0$,} \\
-1, &\text{ else,}
\end{cases}
\]
and $\bx$ be a tree of depth $d$ that is shattered by $\calF$.
Then $x_t = \bx_k(\beps)$ for any $t$ that belongs to epoch $k$.
This concludes the construction of the environment. \\

\begin{enumerate}[leftmargin=*,align=left]
\vspace{5pt}
\item (\blue{2pts}) 
For any online learner, let $s_1, \ldots, s_n \in \cbr{-1, +1}$ be its sequential predictions for $x_1, \ldots, x_n$ in this environment. Calculate the learner's expected loss $\E\sbr{\sum_{t=1}^n \one\cbr{s_t \neq y_t}}$, where the expectation is with respect to the randomness of both the learner and the environment. \\



\vspace{5pt}
\item (\blue{4pts}) 
Calculate $\E\sbr{\inf_{f\in\calF} \sum_{t=1}^n \one\cbr{f(x_t) \neq y_t}}$, the expected loss of the best classifier in $\calF$, where the randomness is with respect to the randomness of the environment. \\


\vspace{5pt}
\item (\blue{4pts}) 
Conclude the statement $\seqV(\calF, n) \geq \sqrt{\frac{d}{8n}}$. Hint: use the Khinchine inequality that says the expected magnitude of the sum of $m$ i.i.d. Rademacher random variables is at least $\sqrt{m/2}$ for any $m \geq 1$.

\end{enumerate}


\end{enumerate}
\end{document}