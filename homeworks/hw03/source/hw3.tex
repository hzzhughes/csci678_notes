\documentclass{article}
\usepackage[final]{nips}

\usepackage{amsthm}
\usepackage{amsmath}
\usepackage{amssymb}
\usepackage{graphicx}
\usepackage{mathtools}
\usepackage{enumerate}
\usepackage{enumitem}
\usepackage{footnote}
\usepackage{float}
\usepackage{xspace}
\usepackage{multirow}
\usepackage{nicefrac}
\usepackage{wrapfig}
\usepackage{framed}
\usepackage{url}
\usepackage[colorlinks=true, linkcolor=blue, citecolor=blue]{hyperref}
\usepackage[ruled, vlined]{algorithm2e}
\usepackage{blkarray}
\PassOptionsToPackage{round}{natbib}

\usepackage{tikz}
\usetikzlibrary{arrows,chains,matrix,positioning,scopes}

\newcommand{\push}{\hspace{0pt plus 1 filll} }	

\newtheorem{lemma}{Lemma}
\newtheorem{theorem}{Theorem}
\newtheorem{cor}{Corollary}
\newtheorem{remark}{Remark}
\newtheorem{prop}{Proposition}
\newtheorem{property}{Property}
\newtheorem{definition}{Definition}
\newtheorem{assumption}{Assumption}

\newcommand{\calA}{{\mathcal{A}}}
\newcommand{\calH}{{\mathcal{H}}}
\newcommand{\calL}{{\mathcal{L}}}
\newcommand{\calX}{{\mathcal{X}}}
\newcommand{\calY}{{\mathcal{Y}}}
\newcommand{\calS}{{\mathcal{S}}}
\newcommand{\calI}{{\mathcal{I}}}
\newcommand{\calD}{{\mathcal{D}}}
\newcommand{\calK}{{\mathcal{K}}}
\newcommand{\calE}{{\mathcal{E}}}
\newcommand{\calR}{{\mathcal{R}}}
\newcommand{\calT}{{\mathcal{T}}}
\newcommand{\calP}{{\mathcal{P}}}
\newcommand{\calZ}{{\mathcal{Z}}}
\newcommand{\calM}{{\mathcal{M}}}
\newcommand{\calN}{{\mathcal{N}}}
\newcommand{\calF}{{\mathcal{F}}}
\newcommand{\calV}{{\mathcal{V}}}
\newcommand{\calQ}{{\mathcal{Q}}}
\newcommand{\ips}{\wh{r}}
\newcommand{\whpi}{\wh{\pi}}
\newcommand{\whE}{\wh{\E}}
\newcommand{\whV}{\wh{V}}
\newcommand{\reg}{{\mathcal{R}}}
\newcommand{\breg}{{\mathcal{\bar{R}}}}
\newcommand{\hmu}{\wh{\mu}}
\newcommand{\tmu}{\wt{\mu}}
\newcommand{\one}{\boldsymbol{1}}
\newcommand{\loss}{\ell}
\newcommand{\hloss}{\wh{\ell}}
\newcommand{\bloss}{\bar{\ell}}
\newcommand{\tloss}{\wt{\ell}}
\newcommand{\htheta}{\wh{\theta}}

\newcommand{\bs}{{\boldsymbol{s}}}
\newcommand{\bz}{{\boldsymbol{z}}}
\newcommand{\bx}{{\boldsymbol{x}}}
\newcommand{\br}{{\boldsymbol{r}}}
\newcommand{\bX}{{\boldsymbol{X}}}
\newcommand{\bu}{{\boldsymbol{u}}}
\newcommand{\by}{{\boldsymbol{y}}}
\newcommand{\bY}{{\boldsymbol{Y}}}
\newcommand{\bg}{{\boldsymbol{g}}}
\newcommand{\ba}{{\boldsymbol{a}}}
\newcommand{\be}{{\boldsymbol{e}}}
\newcommand{\bq}{{\boldsymbol{q}}}
\newcommand{\bp}{{\boldsymbol{p}}}
\newcommand{\bZ}{{\boldsymbol{Z}}}
\newcommand{\bS}{{\boldsymbol{S}}}
\newcommand{\bw}{{\boldsymbol{w}}}
\newcommand{\bW}{{\boldsymbol{W}}}
\newcommand{\bU}{{\boldsymbol{U}}}
\newcommand{\bv}{{\boldsymbol{v}}}
\newcommand{\bzero}{{\boldsymbol{0}}}
\newcommand{\beps}{{\boldsymbol{\epsilon}}}

\newcommand{\blue}[1]{{\color{blue}#1}}

\DeclareMathOperator*{\argmin}{argmin}
\DeclareMathOperator*{\argmax}{argmax}
%\DeclareMathOperator*{\liminf}{liminf}
%\DeclareMathOperator*{\limsup}{limsup}
\DeclareMathOperator*{\range}{range}
\DeclareMathOperator*{\mydet}{det_{+}}

\newcommand{\field}[1]{\mathbb{#1}}
\newcommand{\fY}{\field{Y}}
\newcommand{\fX}{\field{X}}
\newcommand{\fH}{\field{H}}
\newcommand{\fR}{\field{R}}
\newcommand{\fB}{\field{B}}
\newcommand{\fS}{\field{S}}
\newcommand{\fN}{\field{N}}
\newcommand{\E}{\field{E}}
\renewcommand{\P}{\field{P}}

\newcommand{\theset}[2]{ \left\{ {#1} \,:\, {#2} \right\} }
%\newcommand{\inner}[1]{ \left\langle {#1} \right\rangle }
\newcommand{\Ind}[1]{ \field{I}{\{{#1}\}} }
\newcommand{\eye}[1]{ \boldsymbol{I}_{#1} }
\newcommand{\norm}[1]{\left\|{#1}\right\|}
%\newcommand{\trace}[1]{\text{tr}\left({#1}\right)}
\newcommand{\trace}[1]{\textsc{tr}({#1})}
\newcommand{\diag}[1]{\mathrm{diag}\!\left\{{#1}\right\}}
\newcommand{\RE}{{\text{\rm RE}}}
\newcommand{\KL}{{\text{\rm KL}}}
\newcommand{\LCB}{{\text{\rm LCB}}}
\newcommand{\Reg}{{\text{\rm Reg}}}
\newcommand{\Rel}{{\text{\rm Rel}}}
%\newcommand{\ERM}{{\text{\rm ERM}}\xspace}

\newcommand{\defeq}{\stackrel{\rm def}{=}}
\newcommand{\sgn}{\mbox{\sc sgn}}
\newcommand{\scI}{\mathcal{I}}
\newcommand{\scO}{\mathcal{O}}
\newcommand{\scN}{\mathcal{N}}

\newcommand{\dt}{\displaystyle}
\renewcommand{\ss}{\subseteq}
\newcommand{\wh}{\widehat}
\newcommand{\wt}{\widetilde}
\newcommand{\ve}{\varepsilon}
\newcommand{\hlambda}{\wh{\lambda}}
\newcommand{\yhat}{\wh{y}}
\newcommand{\pred}{\yhat}

\newcommand{\hDelta}{\wh{\Delta}}
\newcommand{\hdelta}{\wh{\delta}}
\newcommand{\spin}{\{-1,+1\}}

\newcommand{\paren}[1]{\left({#1}\right)}
\newcommand{\brackets}[1]{\left[{#1}\right]}
\newcommand{\braces}[1]{\left\{{#1}\right\}}

\newcommand{\normt}[1]{\norm{#1}_{t}}
\newcommand{\dualnormt}[1]{\norm{#1}_{t,*}}

\newcommand{\order}{\ensuremath{\mathcal{O}}}
\newcommand{\otil}{\ensuremath{\widetilde{\mathcal{O}}}}
\newcommand{\risk}{{\text{\rm Risk}}}
\newcommand{\iid}{{\text{\rm iid}}}
\newcommand{\seq}{{\text{\rm seq}}}
\newcommand{\iidV}{\calV^\iid}
\newcommand{\seqV}{\calV^\seq}
\newcommand{\poly}{{\text{\rm poly}}}
\newcommand{\sign}{{\text{\rm sign}}}
\newcommand{\ERM}{\pred_{{\text{\rm ERM}}}}
\newcommand{\iidRad}{\calR^{\iid}}
\newcommand{\iidCRad}{\wh{\calR}^{\iid}}
\newcommand{\seqRad}{\calR^{\seq}}
\newcommand{\seqCRad}{\wh{\calR}^{\seq}}
\newcommand{\VC}{{\text{\rm VCdim}}}
\newcommand{\Pdim}{{\text{\rm Pdim}}}
\newcommand{\Ldim}{{\text{\rm Ldim}}}
\newcommand{\fat}{{\text{\rm fat}}}
\newcommand{\sfat}{{\text{\rm sfat}}}
\newcommand{\vol}{{\text{\rm Vol}}}
\newcommand{\Holder}{{H{\"o}lder}\xspace}

%%%%  brackets
\newcommand{\inner}[2]{\left\langle #1,#2 \right\rangle}
\newcommand{\minimax}[1]{\left\llangle #1 \right\rrangle}
\newcommand{\rbr}[1]{\left(#1\right)}
\newcommand{\sbr}[1]{\left[#1\right]}
\newcommand{\cbr}[1]{\left\{#1\right\}}
\newcommand{\nbr}[1]{\left\|#1\right\|}
\newcommand{\abr}[1]{\left|#1\right|}

\DeclareFontFamily{OMX}{MnSymbolE}{}
\DeclareFontShape{OMX}{MnSymbolE}{m}{n}{
    <-6>  MnSymbolE5
   <6-7>  MnSymbolE6
   <7-8>  MnSymbolE7
   <8-9>  MnSymbolE8
   <9-10> MnSymbolE9
  <10-12> MnSymbolE10
  <12->   MnSymbolE12}{}
\DeclareSymbolFont{mnlargesymbols}{OMX}{MnSymbolE}{m}{n}
\SetSymbolFont{mnlargesymbols}{bold}{OMX}{MnSymbolE}{b}{n}
\DeclareMathDelimiter{\llangle}{\mathopen}{mnlargesymbols}{'164}{mnlargesymbols}{'164}
\DeclareMathDelimiter{\rrangle}{\mathclose}{mnlargesymbols}{'171}{mnlargesymbols}{'171}

\usepackage{prettyref}
\newcommand{\pref}[1]{\prettyref{#1}}
\newcommand{\pfref}[1]{Proof of \prettyref{#1}}
\newcommand{\savehyperref}[2]{\texorpdfstring{\hyperref[#1]{#2}}{#2}}
\newrefformat{eq}{\savehyperref{#1}{\textup{(\ref*{#1})}}}
\newrefformat{eqn}{\savehyperref{#1}{Equation~\eqref{#1}}}
\newrefformat{lem}{\savehyperref{#1}{Lemma~\ref*{#1}}}
\newrefformat{def}{\savehyperref{#1}{Definition~\ref*{#1}}}
\newrefformat{line}{\savehyperref{#1}{line~\ref*{#1}}}
\newrefformat{thm}{\savehyperref{#1}{Theorem~\ref*{#1}}}
\newrefformat{corr}{\savehyperref{#1}{Corollary~\ref*{#1}}}
\newrefformat{cor}{\savehyperref{#1}{Corollary~\ref*{#1}}}
\newrefformat{sec}{\savehyperref{#1}{Section~\ref*{#1}}}
\newrefformat{app}{\savehyperref{#1}{Appendix~\ref*{#1}}}
\newrefformat{assum}{\savehyperref{#1}{Assumption~\ref*{#1}}}
\newrefformat{ex}{\savehyperref{#1}{Example~\ref*{#1}}}
\newrefformat{fig}{\savehyperref{#1}{Figure~\ref*{#1}}}
\newrefformat{alg}{\savehyperref{#1}{Algorithm~\ref*{#1}}}
\newrefformat{rem}{\savehyperref{#1}{Remark~\ref*{#1}}}
\newrefformat{conj}{\savehyperref{#1}{Conjecture~\ref*{#1}}}
\newrefformat{prop}{\savehyperref{#1}{Proposition~\ref*{#1}}}
\newrefformat{proto}{\savehyperref{#1}{Protocol~\ref*{#1}}}
\newrefformat{prob}{\savehyperref{#1}{Problem~\ref*{#1}}}
\newrefformat{claim}{\savehyperref{#1}{Claim~\ref*{#1}}}
\newrefformat{que}{\savehyperref{#1}{Question~\ref*{#1}}}

\title{CSCI 678: Theoretical Machine Learning \\ Homework 3 \\ {\small Fall 2024, Instructor: Haipeng Luo}}   

\begin{document}
\maketitle

\textit{This homework is due on {\bf 11/03, 11:59pm}. 
See course website for more instructions on finishing and submitting your homework as well as the late policy. Total points: \blue{40}}\\


\begin{enumerate}[leftmargin=*,align=left]
\item
({\bf Hedge}) (\blue{6pts}) 
For a finite class of binary classifier $\calF \subset \cbr{-1,+1}^\calX$,
under the realizable assumption $\inf_{f\in\calF} \sum_{t=1}^n \one\cbr{f(x_t) \neq y_t} = 0$,
prove that Hedge with learning rate $\eta=1/2$ makes at most $4\ln|\calF|$ mistakes in expectation. 
Hint: use Lemma 1 of Lecture 6.
(Note that this is similar to the guarantee of Halving, but achieved via a proper algorithm this time.)

\newpage
\item 
({\bf Perceptron and sequential fat-shattering dimension})
Recall the sequential fat-shattering dimension $\sfat(\calF, \alpha)$ defined in Lectures 6.
Let $\calX = B^d_2$ and $\calF = \cbr{f_\theta(x) = \inner{\theta}{x} \;|\; \theta \in B^d_2}$.
In this exercise, you will prove $\sfat(\calF, \alpha) \leq \frac{16}{\alpha^2}$ (which is independent of $d$) for any $\alpha > 0$, using an indirect approach that leverages the guarantee of the Perceptron algorithm.

More specifically, suppose that $\bx$ is a $\calX$-valued tree of depth $n$ that is $\alpha$-shattered by $\calF$, with witness $\by$, a $[-1,+1]$-valued tree.
Now, imagine running Perceptron in the following problem instance in $\fR^{d+1}$:

\begin{figure}[H]
\begin{framed}
Let $\theta' = \bzero \in \fR^{d+1}$. For $t = 1, \ldots, n$:
\begin{itemize}
\item Environment reveals example $x'_t = \frac{1}{\sqrt{2}}(\bx_t(y'_{1:t-1}), \by_t(y'_{1:t-1})) \in B_2^{d+1}$.

\item Perceptron algorithm predicts $s_t = \sign(\inner{x_t'}{\theta'})$.

\item Environment reveals $y_t' = -s_t$, forcing Perceptron to make an update $\theta' \leftarrow \theta' + y_t' x_t'$.
\end{itemize}
\end{framed}
\end{figure}

Note that the environment is valid even though it seemingly decides the label $y_t'$ after seeing the algorithm's prediction $s_t$, since Perceptron is a deterministic algorithm (and thus $x_{1:n}'$ and $y_{1:n}'$ are in fact all fixed ahead of time).

\begin{enumerate}[leftmargin=*,align=left]

\vspace{5pt}
\item (\blue{4pts}) 
Prove that the data constructed above satisfy the $\gamma$-margin assumption (Assumption 1 of Lecture 7) with $p=q=2$.
In other words, find a specific value of $\gamma >0$ and show that there exists $\theta_\star' \in B_2^{d+1}$ such that $y_t'\inner{\theta_\star'}{x_t'} \geq \gamma$ holds for all $t = 1, \ldots, n$. \\


\vspace{5pt}
\item (\blue{3pts}) 
Use the guarantee of Perceptron (that is, Theorem 3 of Lecture 7) to conclude $\sfat(\calF, \alpha) \leq \frac{16}{\alpha^2}$.  \\

\end{enumerate}

\newpage
\item 
({\bf Winnow})
When the $\gamma$-margin assumption holds with $p=q=2$, we have seen that Perceptron makes at most $\frac{1}{\gamma^2}$ mistakes for an online binary classification problem.
In this exercise, you will prove a similar result when the $\gamma$-margin assumption holds with $p=1$ and $q=\infty$, using a different algorithm called {\it Winnow}.
To show this, we first consider the following generalization of Perceptron, defined in terms of some \textit{link function} $g: \fR^d \rightarrow \fR^d$.

\begin{minipage}{\linewidth}
\begin{algorithm}[H]
\caption{A generalization of Perceptron}
\label{alg:generalized_Perceptron}
Let $\theta = \bzero$. For $t = 1, \ldots, n$:
\begin{itemize}
\item Receive $x_t$ and predict $s_t = \sign(\inner{x_t}{g(\theta)})$.
\item Receive $y_t\in\cbr{-1,+1}$. If $y_t \neq s_t$, update $\theta \leftarrow \theta + y_t x_t$.
\end{itemize}
\end{algorithm}
\end{minipage}

It is clear that when instantiated with $g$ being the identity mapping $g(\theta) = \theta$, \pref{alg:generalized_Perceptron} is exactly the Perceptron algorithm.
Below, we will see that the Winnow algorithm is also an instance of \pref{alg:generalized_Perceptron} but with a different link function.
Throughout, we assume $x_t \in B_\infty^d$, that is, $\norm{x_t}_\infty \leq 1$, for all $t$.

\begin{enumerate}[leftmargin=*,align=left]
\vspace{5pt}
\item 
Consider running \pref{alg:generalized_Perceptron} with link function $g(\theta) = \exp(\eta\theta)$ and some parameter $\eta > 0$ (where the exponentiation is applied coordinate-wise to the vector $\eta\theta$).
Let's call this the simplified Winnow algorithm.

\begin{enumerate}[leftmargin=*,align=left]
\vspace{5pt}
\item (\blue{4pts}) Find a sequence of loss vectors $\loss_1, \ldots, \loss_n \in [-1,+1]^d$ such that the prediction of simplified Winnow $s_t = \sign(\inner{x_t}{g(\theta)})$ can be equivalently written as $s_t = \sign(\inner{x_t}{p_t})$,  where $p_t \in \Delta(d)$ is a distribution such that 
\[
p_t(i) \propto \exp\rbr{-\eta \sum_{\tau<t} \loss_\tau(i)}, \quad\text{for all $i = 1, \ldots, d$.}
\]


\vspace{5pt}
\item (\blue{8pts})
Based on the reformulation of the last question, apply Lemma 1 of Lecture 6 to show that as long as $\eta \leq 1$, we have for any $\theta^\star \in \Delta(d)$:
\[
\sum_{t=1}^n \one\cbr{y_t \neq s_t} y_t\inner{\theta^\star}{x_t}
\leq \frac{\ln d}{\eta} + \eta M,
\]
where $M = \sum_{t=1}^n \one\cbr{y_t \neq s_t}$ is the total number of mistakes made by the simplified Winnow algorithm. \\


\vspace{5pt}
\item (\blue{3pts})
Consider the following assumption that is slightly stronger than the original $\gamma$-margin assumption with $p=1$ and $q=\infty$:
\begin{equation}\label{eqn:simplified_margin}
\text{there exists $\theta^\star \in \Delta(d)$ such that $y_t\inner{\theta^\star}{x_t} \geq \gamma$ for all $t$.}
\end{equation}
Prove that under this assumption, the total number of mistakes $M$ made by the simplified Winnow algorithm is at most $\frac{4\ln d}{\gamma^2}$ when $\eta = \frac{\gamma}{2} \leq 1$. \\


\end{enumerate}


\vspace{5pt}
\item
Now consider the original $\gamma$-margin assumption, that is: 
\begin{equation}\label{eqn:margin}
\text{there exists $\theta^\star \in B_1^d$ such that $y_t\inner{\theta^\star}{x_t} \geq \gamma$ for all $t$.}
\end{equation}
To deal with this more general case, we will run \pref{alg:generalized_Perceptron} using a different link function $g(\theta) = \exp(\eta\theta)-\exp(-\eta\theta)$ (again, the exponentiation is coordinate-wise).
This is the (actual) Winnow algorithm.
\\

\begin{enumerate}[leftmargin=*,align=left]
\vspace{5pt}
\item (\blue{4pts}) 
Prove that the Winnow algorithm is the same as running the simplified Winnow algorithm over examples $x_t' = (x_t, -x_t) \in B_\infty^{2d}$ and $y_t' = y_t$ for $t = 1, \ldots, n$. \\


\vspace{5pt}
\item (\blue{6pts}) 
Under the margin assumption \pref{eqn:margin}, further prove that the examples $(x_{1:n}', y_{1:n}')$ defined above satisfy \pref{eqn:simplified_margin} for some margin $\gamma'$, that is, there exists $\theta' \in \Delta(2d)$ such that $y_t'\inner{\theta'}{x_t} \geq \gamma'$ for all $t$. \\


\vspace{5pt}
\item (\blue{2pts}) 
Finally, under the margin assumption  \pref{eqn:margin}, use the result from Question (a)iii to provide a bound on the total number of mistakes made by the Winnow algorithm when $\eta = \frac{\gamma}{2}$. \\


\end{enumerate}

\end{enumerate}


\end{enumerate}
\end{document}