\documentclass{article}
\usepackage[final]{nips}

\usepackage{amsthm}
\usepackage{amsmath}
\usepackage{amssymb}
\usepackage{graphicx}
\usepackage{mathtools}
\usepackage{enumerate}
\usepackage{enumitem}
\usepackage{footnote}
\usepackage{float}
\usepackage{xspace}
\usepackage{multirow}
\usepackage{nicefrac}
\usepackage{wrapfig}
\usepackage{framed}
\usepackage{url}
\usepackage[colorlinks=true, linkcolor=blue, citecolor=blue]{hyperref}
\usepackage[ruled, vlined]{algorithm2e}
\usepackage{blkarray}
\PassOptionsToPackage{round}{natbib}

\usepackage{tikz}
\usetikzlibrary{arrows,chains,matrix,positioning,scopes}

\newcommand{\push}{\hspace{0pt plus 1 filll} }	

\newtheorem{lemma}{Lemma}
\newtheorem{theorem}{Theorem}
\newtheorem{cor}{Corollary}
\newtheorem{remark}{Remark}
\newtheorem{prop}{Proposition}
\newtheorem{property}{Property}
\newtheorem{definition}{Definition}
\newtheorem{assumption}{Assumption}

\newcommand{\calA}{{\mathcal{A}}}
\newcommand{\calH}{{\mathcal{H}}}
\newcommand{\calL}{{\mathcal{L}}}
\newcommand{\calX}{{\mathcal{X}}}
\newcommand{\calY}{{\mathcal{Y}}}
\newcommand{\calS}{{\mathcal{S}}}
\newcommand{\calI}{{\mathcal{I}}}
\newcommand{\calD}{{\mathcal{D}}}
\newcommand{\calK}{{\mathcal{K}}}
\newcommand{\calE}{{\mathcal{E}}}
\newcommand{\calR}{{\mathcal{R}}}
\newcommand{\calT}{{\mathcal{T}}}
\newcommand{\calP}{{\mathcal{P}}}
\newcommand{\calZ}{{\mathcal{Z}}}
\newcommand{\calM}{{\mathcal{M}}}
\newcommand{\calN}{{\mathcal{N}}}
\newcommand{\calF}{{\mathcal{F}}}
\newcommand{\calV}{{\mathcal{V}}}
\newcommand{\calQ}{{\mathcal{Q}}}
\newcommand{\ips}{\wh{r}}
\newcommand{\whpi}{\wh{\pi}}
\newcommand{\whE}{\wh{\E}}
\newcommand{\whV}{\wh{V}}
\newcommand{\reg}{{\mathcal{R}}}
\newcommand{\breg}{{\mathcal{\bar{R}}}}
\newcommand{\hmu}{\wh{\mu}}
\newcommand{\tmu}{\wt{\mu}}
\newcommand{\one}{\boldsymbol{1}}
\newcommand{\loss}{\ell}
\newcommand{\hloss}{\wh{\ell}}
\newcommand{\bloss}{\bar{\ell}}
\newcommand{\tloss}{\wt{\ell}}
\newcommand{\htheta}{\wh{\theta}}

\newcommand{\bs}{{\boldsymbol{s}}}
\newcommand{\bz}{{\boldsymbol{z}}}
\newcommand{\bx}{{\boldsymbol{x}}}
\newcommand{\br}{{\boldsymbol{r}}}
\newcommand{\bX}{{\boldsymbol{X}}}
\newcommand{\bu}{{\boldsymbol{u}}}
\newcommand{\by}{{\boldsymbol{y}}}
\newcommand{\bY}{{\boldsymbol{Y}}}
\newcommand{\bg}{{\boldsymbol{g}}}
\newcommand{\ba}{{\boldsymbol{a}}}
\newcommand{\be}{{\boldsymbol{e}}}
\newcommand{\bq}{{\boldsymbol{q}}}
\newcommand{\bp}{{\boldsymbol{p}}}
\newcommand{\bZ}{{\boldsymbol{Z}}}
\newcommand{\bS}{{\boldsymbol{S}}}
\newcommand{\bw}{{\boldsymbol{w}}}
\newcommand{\bW}{{\boldsymbol{W}}}
\newcommand{\bU}{{\boldsymbol{U}}}
\newcommand{\bv}{{\boldsymbol{v}}}
\newcommand{\bzero}{{\boldsymbol{0}}}
\newcommand{\beps}{{\boldsymbol{\epsilon}}}

\newcommand{\blue}[1]{{\color{blue}#1}}

\DeclareMathOperator*{\argmin}{argmin}
\DeclareMathOperator*{\argmax}{argmax}
%\DeclareMathOperator*{\liminf}{liminf}
%\DeclareMathOperator*{\limsup}{limsup}
\DeclareMathOperator*{\range}{range}
\DeclareMathOperator*{\mydet}{det_{+}}

\newcommand{\field}[1]{\mathbb{#1}}
\newcommand{\fY}{\field{Y}}
\newcommand{\fX}{\field{X}}
\newcommand{\fH}{\field{H}}
\newcommand{\fR}{\field{R}}
\newcommand{\fB}{\field{B}}
\newcommand{\fS}{\field{S}}
\newcommand{\fN}{\field{N}}
\newcommand{\E}{\field{E}}
\renewcommand{\P}{\field{P}}

\newcommand{\theset}[2]{ \left\{ {#1} \,:\, {#2} \right\} }
%\newcommand{\inner}[1]{ \left\langle {#1} \right\rangle }
\newcommand{\Ind}[1]{ \field{I}{\{{#1}\}} }
\newcommand{\eye}[1]{ \boldsymbol{I}_{#1} }
\newcommand{\norm}[1]{\left\|{#1}\right\|}
%\newcommand{\trace}[1]{\text{tr}\left({#1}\right)}
\newcommand{\trace}[1]{\textsc{tr}({#1})}
\newcommand{\diag}[1]{\mathrm{diag}\!\left\{{#1}\right\}}
\newcommand{\RE}{{\text{\rm RE}}}
\newcommand{\KL}{{\text{\rm KL}}}
\newcommand{\LCB}{{\text{\rm LCB}}}
\newcommand{\Reg}{{\text{\rm Reg}}}
\newcommand{\Rel}{{\text{\rm Rel}}}
%\newcommand{\ERM}{{\text{\rm ERM}}\xspace}

\newcommand{\defeq}{\stackrel{\rm def}{=}}
\newcommand{\sgn}{\mbox{\sc sgn}}
\newcommand{\scI}{\mathcal{I}}
\newcommand{\scO}{\mathcal{O}}
\newcommand{\scN}{\mathcal{N}}

\newcommand{\dt}{\displaystyle}
\renewcommand{\ss}{\subseteq}
\newcommand{\wh}{\widehat}
\newcommand{\wt}{\widetilde}
\newcommand{\ve}{\varepsilon}
\newcommand{\hlambda}{\wh{\lambda}}
\newcommand{\yhat}{\wh{y}}
\newcommand{\pred}{\yhat}

\newcommand{\hDelta}{\wh{\Delta}}
\newcommand{\hdelta}{\wh{\delta}}
\newcommand{\spin}{\{-1,+1\}}

\newcommand{\paren}[1]{\left({#1}\right)}
\newcommand{\brackets}[1]{\left[{#1}\right]}
\newcommand{\braces}[1]{\left\{{#1}\right\}}

\newcommand{\normt}[1]{\norm{#1}_{t}}
\newcommand{\dualnormt}[1]{\norm{#1}_{t,*}}

\newcommand{\order}{\ensuremath{\mathcal{O}}}
\newcommand{\otil}{\ensuremath{\widetilde{\mathcal{O}}}}
\newcommand{\risk}{{\text{\rm Risk}}}
\newcommand{\iid}{{\text{\rm iid}}}
\newcommand{\seq}{{\text{\rm seq}}}
\newcommand{\iidV}{\calV^\iid}
\newcommand{\seqV}{\calV^\seq}
\newcommand{\poly}{{\text{\rm poly}}}
\newcommand{\sign}{{\text{\rm sign}}}
\newcommand{\ERM}{\pred_{{\text{\rm ERM}}}}
\newcommand{\iidRad}{\calR^{\iid}}
\newcommand{\iidCRad}{\wh{\calR}^{\iid}}
\newcommand{\seqRad}{\calR^{\seq}}
\newcommand{\seqCRad}{\wh{\calR}^{\seq}}
\newcommand{\VC}{{\text{\rm VCdim}}}
\newcommand{\Pdim}{{\text{\rm Pdim}}}
\newcommand{\Ldim}{{\text{\rm Ldim}}}
\newcommand{\fat}{{\text{\rm fat}}}
\newcommand{\sfat}{{\text{\rm sfat}}}
\newcommand{\vol}{{\text{\rm Vol}}}
\newcommand{\Holder}{{H{\"o}lder}\xspace}

%%%%  brackets
\newcommand{\inner}[2]{\left\langle #1,#2 \right\rangle}
\newcommand{\minimax}[1]{\left\llangle #1 \right\rrangle}
\newcommand{\rbr}[1]{\left(#1\right)}
\newcommand{\sbr}[1]{\left[#1\right]}
\newcommand{\cbr}[1]{\left\{#1\right\}}
\newcommand{\nbr}[1]{\left\|#1\right\|}
\newcommand{\abr}[1]{\left|#1\right|}

\DeclareFontFamily{OMX}{MnSymbolE}{}
\DeclareFontShape{OMX}{MnSymbolE}{m}{n}{
    <-6>  MnSymbolE5
   <6-7>  MnSymbolE6
   <7-8>  MnSymbolE7
   <8-9>  MnSymbolE8
   <9-10> MnSymbolE9
  <10-12> MnSymbolE10
  <12->   MnSymbolE12}{}
\DeclareSymbolFont{mnlargesymbols}{OMX}{MnSymbolE}{m}{n}
\SetSymbolFont{mnlargesymbols}{bold}{OMX}{MnSymbolE}{b}{n}
\DeclareMathDelimiter{\llangle}{\mathopen}{mnlargesymbols}{'164}{mnlargesymbols}{'164}
\DeclareMathDelimiter{\rrangle}{\mathclose}{mnlargesymbols}{'171}{mnlargesymbols}{'171}

\usepackage{prettyref}
\newcommand{\pref}[1]{\prettyref{#1}}
\newcommand{\pfref}[1]{Proof of \prettyref{#1}}
\newcommand{\savehyperref}[2]{\texorpdfstring{\hyperref[#1]{#2}}{#2}}
\newrefformat{eq}{\savehyperref{#1}{\textup{(\ref*{#1})}}}
\newrefformat{eqn}{\savehyperref{#1}{Equation~\eqref{#1}}}
\newrefformat{lem}{\savehyperref{#1}{Lemma~\ref*{#1}}}
\newrefformat{def}{\savehyperref{#1}{Definition~\ref*{#1}}}
\newrefformat{line}{\savehyperref{#1}{line~\ref*{#1}}}
\newrefformat{thm}{\savehyperref{#1}{Theorem~\ref*{#1}}}
\newrefformat{corr}{\savehyperref{#1}{Corollary~\ref*{#1}}}
\newrefformat{cor}{\savehyperref{#1}{Corollary~\ref*{#1}}}
\newrefformat{sec}{\savehyperref{#1}{Section~\ref*{#1}}}
\newrefformat{app}{\savehyperref{#1}{Appendix~\ref*{#1}}}
\newrefformat{assum}{\savehyperref{#1}{Assumption~\ref*{#1}}}
\newrefformat{ex}{\savehyperref{#1}{Example~\ref*{#1}}}
\newrefformat{fig}{\savehyperref{#1}{Figure~\ref*{#1}}}
\newrefformat{alg}{\savehyperref{#1}{Algorithm~\ref*{#1}}}
\newrefformat{rem}{\savehyperref{#1}{Remark~\ref*{#1}}}
\newrefformat{conj}{\savehyperref{#1}{Conjecture~\ref*{#1}}}
\newrefformat{prop}{\savehyperref{#1}{Proposition~\ref*{#1}}}
\newrefformat{proto}{\savehyperref{#1}{Protocol~\ref*{#1}}}
\newrefformat{prob}{\savehyperref{#1}{Problem~\ref*{#1}}}
\newrefformat{claim}{\savehyperref{#1}{Claim~\ref*{#1}}}
\newrefformat{que}{\savehyperref{#1}{Question~\ref*{#1}}}

\title{CSCI 678: Theoretical Machine Learning \\ Homework 1 \\ {\small Fall 2024, Instructor: Haipeng Luo}}  

\begin{document}
\maketitle
\textit{This homework is due on {\bf 9/22, 11:59pm}. 
See course website for more instructions on finishing and submitting your homework as well as the late policy. Total points: \blue{50}}\\


\begin{enumerate}[leftmargin=*,align=left]
\item
({\bf Rademacher complexity and Dudley entropy integral}) 
Consider inputs $x_1, \ldots, x_n \in \fR^d$ and the linear class $\calF = \cbr{f_\theta(x) = \inner{\theta}{x} \;|\; \theta \in \fR^d, \norm{\theta}_2 \leq b}$.

\begin{enumerate}[leftmargin=*,align=left]
\item (\blue{5pts})
Prove the following:
\[
\iidCRad(\calF; x_{1:n})  \leq \frac{b}{n}\sqrt{\sum_{t=1}^n \|x_t\|_2^2}
\]
using the definition of Rademacher complexity directly (that is, without invoking its upper bounds in terms of covering numbers or other measures).
Hint: you will need to use the inequality $\E\sbr{a} \leq \sqrt{\E\sbr{a^2}}$ for any $a\geq 0$ (which is a consequence of Jensen's inequality). \\



\newpage
\item (\blue{3pts}) 
In Lecture 4, we will prove the following log covering number bound for this class: $\ln \calN_2(\calF|_{x_{1:n}}, \alpha) \leq \frac{b^2\ln (2d)\sum_{t=1}^n \norm{x_t}_2^2}{n\alpha^2}$.
Use this bound and the Dudley entropy integral to prove
\[
\iidCRad(\calF; x_{1:n})  \leq \otil\rbr{\frac{b}{n}\sqrt{\sum_{t=1}^n \|x_t\|_2^2}},
\]
where the $\otil(\cdot)$ notation hides all logarithmic factors.
(This bound is thus of the same order as the one from the last question.)


\end{enumerate}

\newpage
\item
({\bf Growth function and VC-dimension})
\begin{enumerate}[leftmargin=*,align=left]
\vspace{5pt}
\item 
Let $\calX = \fR^d$ and $\calF = \cbr{f_{\theta,b}(x) =  \sign\rbr{\inner{x}{\theta}+b} \;|\; \theta \in \fR^d, b \in \fR}$ be the set of $d$-dimensional linear classifiers.
Prove $\VC(\calF) = d+1$ by following the two steps below.

\begin{enumerate}[leftmargin=*,align=left]
\vspace{5pt}
\item (\blue{4pts}) 
Construct $d+1$ points $x_1, \ldots, x_{d+1} \in \fR^d$ and argue that for any labeling $y_1, \ldots, y_{d+1} \in \cbr{-1,+1}$, there exists $f \in \calF$ such that $f(x_t) = y_t$ for all $t = 1, \ldots, d+1$. \\


\vspace{5pt}
\item (\blue{6pts}) 
Prove that for any $d+2$ points $x_1, \ldots, x_{d+2} \in \fR^d$, there exists a labeling $y_1, \ldots, y_{d+2} \in \cbr{-1,+1}$ such that no $f\in\calF$ satisfies $f(x_t) = y_t$ for all $t = 1, \ldots, d+2$. 
Hint: consider appending 1 to the end of each of the $d+2$ points: $(x_1, 1), \cdots, (x_{d+2}, 1) \in\fR^{d+1}$, and start with the fact that these $d+2$ points must be linearly dependent (since they live in $\fR^{d+1}$).
\\


\end{enumerate}

\newpage
\vspace{5pt}
\item (\blue{5pts}) 
Let $\calX = \fR$ and $\calF = \cbr{f_\theta(x) = \sign(\sin(\theta x)) \;|\; \theta \in \fR}$.
Prove that for any $n$, if $x_t = 2^{-2t}$, then $\calF$ shatters the set $x_{1:n}$,
which means $\VC(\calF) = \infty$.
(Hint: for any labeling $y_{1:n}$, consider $\theta = \pi\rbr{1+ \sum_{i=1}^n (1-y_i)2^{2i-1}}$.) \\

\end{enumerate}
\newpage


\item ({\bf Covering number})
\begin{enumerate}[leftmargin=*,align=left]
\vspace{5pt}
\item In Proposition 2 of Lecture 3, via a volumetric argument we show that the linear class $\calF = \cbr{f_\theta(x) = \inner{\theta}{x} \;|\; \theta \in B^d_p}$ for $\calX = B^d_q$ and some $p\geq 1$ and $q\geq 1$ such that $\frac{1}{p}+\frac{1}{q}=1$ has bounded pointwise covering number: $\calN(\calF, \alpha) \leq \rbr{\frac{2}{\alpha}+1}^d$ for any $0\leq \alpha \leq 1$.
Follow the two steps below to further show $\calN(\calF, \alpha) \geq \rbr{\frac{1}{2\alpha}}^d$.

\begin{enumerate}[leftmargin=*,align=left]
\vspace{5pt}
\item (\blue{5pts}) 
Given any pointwise $\alpha$-cover $\calH \subset [-1,+1]^\calX$,
construct a pointwise $2\alpha$-cover $\calH' \subset \calF$ so that $|\calH'| \leq |\calH|$ (note that $\calH'$ has to be a subset of $\calF$). \\


\vspace{5pt}
\item (\blue{6pts}) 
Prove that if $\calH' \subset \calF$ is a pointwise $2\alpha$-cover of $\calF$, then we must have $|\calH'| \geq \rbr{\frac{1}{2\alpha}}^d$, which then implies $\calN(\calF, \alpha) \geq \rbr{\frac{1}{2\alpha}}^d$ as desired.  Hint: use a similar volumetric argument.\\


\end{enumerate}

\newpage
\item Let $v_1, \ldots, v_d \in B_2^n$ be $d$ points within the $n$-dimensional $\ell_2$-norm unit ball and 
\[
\calS = \cbr{\sum_{i=1}^d \beta_i v_i \;\bigg\rvert\; \beta_i \geq 0, \;\forall i, \text{ and }\sum_{i=1}^d \beta_i \leq B }
\] 
be the convex hull of these $d$ points scaled by $B > 0$.

\begin{enumerate}[leftmargin=*,align=left]
\vspace{5pt}
\item (\blue{5pts}) 
Prove $\calN_2(\calS, \alpha) \leq \rbr{\frac{2B}{\sqrt{n}\alpha}+1}^d$. \\


\newpage
\vspace{5pt}
\item Follow the steps below to prove a different covering number bound $\calN_2(\calS, \alpha) \leq d^{\frac{B^2}{n\alpha^2}}$.


\begin{enumerate}[leftmargin=*,align=left]
\vspace{5pt}
\item (\blue{4pts}) 
For any $v = \sum_{i=1}^d \beta_i v_i \in \calS$, let $\beta = (\beta_1, \ldots, \beta_d)$ and define $m$ i.i.d. random variables $u_1, \ldots, u_m$, each of which is $\norm{\beta}_1 v_i$ with probability $\beta_i / \norm{\beta}_1$ for $i = 1, \ldots, d$.
Prove that the mean of these random variables is $v$ and the variance of $u = \frac{1}{m} \sum_{j=1}^m u_j$ is bounded as:
\[
\E\sbr{\norm{u - v}_2^2} \leq \frac{\norm{\beta}_1^2}{m}.
\]


\vspace{5pt}
\item (\blue{7pts}) Prove that the following is an $\alpha$-cover of $\calS$ with respect to $\ell_2$-norm:
\[
\calS' = \cbr{\frac{B}{M}\sum_{i=1}^d m_i v_i \;\bigg\rvert\; \text{each $m_i$ is a nonnegative integer and $\sum_{i=1}^d m_i \leq M$}}
\]
where $M = \frac{B^2}{n\alpha^2}$. 
(The statement $\calN_2(\calS, \alpha) \leq d^{\frac{B^2}{n\alpha^2}}$ then follows immediately.) \\

\end{enumerate}
\end{enumerate}
\end{enumerate}


\end{enumerate}
\end{document}