\documentclass{article}
\input{header}

\title{CSCI 678: Theoretical Machine Learning \\ Homework 1 \\ {\small Fall 2024, Instructor: Haipeng Luo}}  

\begin{document}
\maketitle
\textit{This homework is due on {\bf 9/22, 11:59pm}. 
See course website for more instructions on finishing and submitting your homework as well as the late policy. Total points: \blue{50}}\\


\begin{enumerate}[leftmargin=*,align=left]
\item
({\bf Rademacher complexity and Dudley entropy integral}) 
Consider inputs $x_1, \ldots, x_n \in \fR^d$ and the linear class $\calF = \cbr{f_\theta(x) = \inner{\theta}{x} \;|\; \theta \in \fR^d, \norm{\theta}_2 \leq b}$.

\begin{enumerate}[leftmargin=*,align=left]
\item (\blue{5pts})
Prove the following:
\[
\iidCRad(\calF; x_{1:n})  \leq \frac{b}{n}\sqrt{\sum_{t=1}^n \|x_t\|_2^2}
\]
using the definition of Rademacher complexity directly (that is, without invoking its upper bounds in terms of covering numbers or other measures).
Hint: you will need to use the inequality $\E\sbr{a} \leq \sqrt{\E\sbr{a^2}}$ for any $a\geq 0$ (which is a consequence of Jensen's inequality). \\



\newpage
\item (\blue{3pts}) 
In Lecture 4, we will prove the following log covering number bound for this class: $\ln \calN_2(\calF|_{x_{1:n}}, \alpha) \leq \frac{b^2\ln (2d)\sum_{t=1}^n \norm{x_t}_2^2}{n\alpha^2}$.
Use this bound and the Dudley entropy integral to prove
\[
\iidCRad(\calF; x_{1:n})  \leq \otil\rbr{\frac{b}{n}\sqrt{\sum_{t=1}^n \|x_t\|_2^2}},
\]
where the $\otil(\cdot)$ notation hides all logarithmic factors.
(This bound is thus of the same order as the one from the last question.)


\end{enumerate}

\newpage
\item
({\bf Growth function and VC-dimension})
\begin{enumerate}[leftmargin=*,align=left]
\vspace{5pt}
\item 
Let $\calX = \fR^d$ and $\calF = \cbr{f_{\theta,b}(x) =  \sign\rbr{\inner{x}{\theta}+b} \;|\; \theta \in \fR^d, b \in \fR}$ be the set of $d$-dimensional linear classifiers.
Prove $\VC(\calF) = d+1$ by following the two steps below.

\begin{enumerate}[leftmargin=*,align=left]
\vspace{5pt}
\item (\blue{4pts}) 
Construct $d+1$ points $x_1, \ldots, x_{d+1} \in \fR^d$ and argue that for any labeling $y_1, \ldots, y_{d+1} \in \cbr{-1,+1}$, there exists $f \in \calF$ such that $f(x_t) = y_t$ for all $t = 1, \ldots, d+1$. \\


\vspace{5pt}
\item (\blue{6pts}) 
Prove that for any $d+2$ points $x_1, \ldots, x_{d+2} \in \fR^d$, there exists a labeling $y_1, \ldots, y_{d+2} \in \cbr{-1,+1}$ such that no $f\in\calF$ satisfies $f(x_t) = y_t$ for all $t = 1, \ldots, d+2$. 
Hint: consider appending 1 to the end of each of the $d+2$ points: $(x_1, 1), \cdots, (x_{d+2}, 1) \in\fR^{d+1}$, and start with the fact that these $d+2$ points must be linearly dependent (since they live in $\fR^{d+1}$).
\\


\end{enumerate}

\newpage
\vspace{5pt}
\item (\blue{5pts}) 
Let $\calX = \fR$ and $\calF = \cbr{f_\theta(x) = \sign(\sin(\theta x)) \;|\; \theta \in \fR}$.
Prove that for any $n$, if $x_t = 2^{-2t}$, then $\calF$ shatters the set $x_{1:n}$,
which means $\VC(\calF) = \infty$.
(Hint: for any labeling $y_{1:n}$, consider $\theta = \pi\rbr{1+ \sum_{i=1}^n (1-y_i)2^{2i-1}}$.) \\

\end{enumerate}
\newpage


\item ({\bf Covering number})
\begin{enumerate}[leftmargin=*,align=left]
\vspace{5pt}
\item In Proposition 2 of Lecture 3, via a volumetric argument we show that the linear class $\calF = \cbr{f_\theta(x) = \inner{\theta}{x} \;|\; \theta \in B^d_p}$ for $\calX = B^d_q$ and some $p\geq 1$ and $q\geq 1$ such that $\frac{1}{p}+\frac{1}{q}=1$ has bounded pointwise covering number: $\calN(\calF, \alpha) \leq \rbr{\frac{2}{\alpha}+1}^d$ for any $0\leq \alpha \leq 1$.
Follow the two steps below to further show $\calN(\calF, \alpha) \geq \rbr{\frac{1}{2\alpha}}^d$.

\begin{enumerate}[leftmargin=*,align=left]
\vspace{5pt}
\item (\blue{5pts}) 
Given any pointwise $\alpha$-cover $\calH \subset [-1,+1]^\calX$,
construct a pointwise $2\alpha$-cover $\calH' \subset \calF$ so that $|\calH'| \leq |\calH|$ (note that $\calH'$ has to be a subset of $\calF$). \\


\vspace{5pt}
\item (\blue{6pts}) 
Prove that if $\calH' \subset \calF$ is a pointwise $2\alpha$-cover of $\calF$, then we must have $|\calH'| \geq \rbr{\frac{1}{2\alpha}}^d$, which then implies $\calN(\calF, \alpha) \geq \rbr{\frac{1}{2\alpha}}^d$ as desired.  Hint: use a similar volumetric argument.\\


\end{enumerate}

\newpage
\item Let $v_1, \ldots, v_d \in B_2^n$ be $d$ points within the $n$-dimensional $\ell_2$-norm unit ball and 
\[
\calS = \cbr{\sum_{i=1}^d \beta_i v_i \;\bigg\rvert\; \beta_i \geq 0, \;\forall i, \text{ and }\sum_{i=1}^d \beta_i \leq B }
\] 
be the convex hull of these $d$ points scaled by $B > 0$.

\begin{enumerate}[leftmargin=*,align=left]
\vspace{5pt}
\item (\blue{5pts}) 
Prove $\calN_2(\calS, \alpha) \leq \rbr{\frac{2B}{\sqrt{n}\alpha}+1}^d$. \\


\newpage
\vspace{5pt}
\item Follow the steps below to prove a different covering number bound $\calN_2(\calS, \alpha) \leq d^{\frac{B^2}{n\alpha^2}}$.


\begin{enumerate}[leftmargin=*,align=left]
\vspace{5pt}
\item (\blue{4pts}) 
For any $v = \sum_{i=1}^d \beta_i v_i \in \calS$, let $\beta = (\beta_1, \ldots, \beta_d)$ and define $m$ i.i.d. random variables $u_1, \ldots, u_m$, each of which is $\norm{\beta}_1 v_i$ with probability $\beta_i / \norm{\beta}_1$ for $i = 1, \ldots, d$.
Prove that the mean of these random variables is $v$ and the variance of $u = \frac{1}{m} \sum_{j=1}^m u_j$ is bounded as:
\[
\E\sbr{\norm{u - v}_2^2} \leq \frac{\norm{\beta}_1^2}{m}.
\]


\vspace{5pt}
\item (\blue{7pts}) Prove that the following is an $\alpha$-cover of $\calS$ with respect to $\ell_2$-norm:
\[
\calS' = \cbr{\frac{B}{M}\sum_{i=1}^d m_i v_i \;\bigg\rvert\; \text{each $m_i$ is a nonnegative integer and $\sum_{i=1}^d m_i \leq M$}}
\]
where $M = \frac{B^2}{n\alpha^2}$. 
(The statement $\calN_2(\calS, \alpha) \leq d^{\frac{B^2}{n\alpha^2}}$ then follows immediately.) \\

\end{enumerate}
\end{enumerate}
\end{enumerate}


\end{enumerate}
\end{document}